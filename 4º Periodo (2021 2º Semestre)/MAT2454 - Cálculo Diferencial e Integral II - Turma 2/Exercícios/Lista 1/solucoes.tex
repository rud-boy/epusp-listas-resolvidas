\documentclass{article}
\usepackage[brazil]{babel}
\usepackage{hyperref}
\usepackage{fancyhdr}
\usepackage[
  a4paper,
  includeheadfoot,
  left = 1cm,
  right = 1cm,
  top = 0.5cm,
  bottom = 0.5cm
]{geometry}

% Limpar estilos para uso de fancyhdr
\pagestyle{empty}

\author{Rudah C. Amaral}
\title{Lista 1 - Cálculo Diferencial e Integral II}
\date{21 de agosto de 2021}

%%%%%%%%%%%%%%%%%%%%%%%%%%%%%%%%%%%% TÍTULO %%%%%%%%%%%%%%%%%%%%%%%%%%%%%%%%%%%%

\begin{document}
\maketitle
\clearpage

%%%%%%%%%%%%%%%%%%%%%%%%%%%%%%%%%%% SUMÁRIO %%%%%%%%%%%%%%%%%%%%%%%%%%%%%%%%%%%%

\tableofcontents
\label{TOC}
\clearpage

%%%%%%%%%%%%%%%%%%%%%%%%%%%%%%%% FIM DO SUMÁRIO %%%%%%%%%%%%%%%%%%%%%%%%%%%%%%%%

% Estilos de fancyhdr
\pagestyle{fancy}
\cfoot{\hyperref[TOC]{Retornar para o sumário}}
\setlength{\headheight}{16pt}
\rfoot{Página \thepage}

%%%%%%%%%%%%%%%%%%%%%% EXERCÍCIOS DE POLINÔMIOS DE TAYLOR %%%%%%%%%%%%%%%%%%%%%%

\section{1. Polinômio de Taylor}

%%%%%%%%%%%%%%%%%%%%%%%%%%%%%%%%% EXERCÍCIO 1 %%%%%%%%%%%%%%%%%%%%%%%%%%%%%%%%%%

\subsection{1.}
Utilizando o polinômio de Taylor de ordem 2, calcule um valor aproximado e
avalie o erro.

\subsubsection{(a)}
$\sqrt[3]{8,2}$
\begin{align}
  P_2(x)
  &= f(x_0) + f'(x_0)(x-x_0) + \frac{f''(x_0)(x-x_0)^2} {2!} \\
  %%%%%%%%%%%%%%%%%%%%%%%%%%%%%%%%%%%%%%%%%%%%%%%%%%%%%%%%%%%%%%%%%%%%%%%%%%%%%%
  E_2(x)
  &= \frac{f'''(\bar{x}) \cdot (x-x_0)^3}{3!} \\
  %%%%%%%%%%%%%%%%%%%%%%%%%%%%%%%%%%%%%%%%%%%%%%%%%%%%%%%%%%%%%%%%%%%%%%%%%%%%%%
  f(x)
  &= \sqrt[3]{x}
  = x^{\frac{1}{3}} \\
  %%%%%%%%%%%%%%%%%%%%%%%%%%%%%%%%%%%%%%%%%%%%%%%%%%%%%%%%%%%%%%%%%%%%%%%%%%%%%%
  f'(x)
  &= \frac{x^{-\frac{2}{3}}}{3}
  = \frac{\sqrt[3]{\frac{1}{x}^2}}{3} \\
  %%%%%%%%%%%%%%%%%%%%%%%%%%%%%%%%%%%%%%%%%%%%%%%%%%%%%%%%%%%%%%%%%%%%%%%%%%%%%%
  f''(x)
  &= \frac{1}{3} \cdot \frac{-2x^{-\frac{5}{3}}}{3}
  = -\frac{2}{9x^{\frac{5}{3}}} \\
  %%%%%%%%%%%%%%%%%%%%%%%%%%%%%%%%%%%%%%%%%%%%%%%%%%%%%%%%%%%%%%%%%%%%%%%%%%%%%%
  f'''(x)
  &= - \frac{2}{9} \cdot \frac{-5x^{-\frac{8}{3}}}{3}
  = \frac{10}{27x^{\frac{8}{3}}}
  && \text{O maior erro possível ocorre quando $\bar{x} = 8$} \\
  %%%%%%%%%%%%%%%%%%%%%%%%%%%%%%%%%%%%%%%%%%%%%%%%%%%%%%%%%%%%%%%%%%%%%%%%%%%%%%
  x_0
  &= 8 \\
  %%%%%%%%%%%%%%%%%%%%%%%%%%%%%%%%%%%%%%%%%%%%%%%%%%%%%%%%%%%%%%%%%%%%%%%%%%%%%%
  f(8)
  &= \sqrt[3]{8}
  = 2 \\
  %%%%%%%%%%%%%%%%%%%%%%%%%%%%%%%%%%%%%%%%%%%%%%%%%%%%%%%%%%%%%%%%%%%%%%%%%%%%%%
  f'(8)
  &= \frac{\sqrt[3]{\frac{1}{8}^2}}{3}
  = \frac{\frac{1}{4}}{3}
  = \frac{1}{12} \\
  %%%%%%%%%%%%%%%%%%%%%%%%%%%%%%%%%%%%%%%%%%%%%%%%%%%%%%%%%%%%%%%%%%%%%%%%%%%%%%
  f''(8)
  &= -\frac{2}{9 \cdot 8^{\frac{5}{3}}}
  = -\frac{2}{9 \cdot \sqrt[3]{8^5}}
  = -\frac{2}{9 \cdot 2^5}
  = -\frac{1}{144} \\
  %%%%%%%%%%%%%%%%%%%%%%%%%%%%%%%%%%%%%%%%%%%%%%%%%%%%%%%%%%%%%%%%%%%%%%%%%%%%%%
  P_2(8,2)
  &= 2 + \frac{8,2 - 8}{12} - \frac{\frac{1}{144}(8,2 - 8)^2}{2}
  = 2 + \frac{\frac{2}{10}}{12} - \frac{\frac{1}{144} \cdot \frac{2^2}{10^2}}{2} \\
  &= 2 + \frac{1}{10 \cdot 6} - \frac{1}{144} \cdot \frac{2}{10^2}
  = 2 + \frac{1}{60} - \frac{1}{7200} \notag \\
  %%%%%%%%%%%%%%%%%%%%%%%%%%%%%%%%%%%%%%%%%%%%%%%%%%%%%%%%%%%%%%%%%%%%%%%%%%%%%%
  |E_2(8,2)|
  &= \left|\frac{\frac{10}{27 \cdot 8^{\frac{8}{3}}} \cdot (8,2 - 8)^3}{3!}\right|
  \leq \frac{\frac{10}{27 \cdot \sqrt[3]{8^8}} \cdot (0,2)^3}{6} \\
  &= \frac{10}{27 \cdot 2^8} \cdot \frac{2^3}{10^3 \cdot 6}
  = \frac{1}{3^4 \cdot 2^6 \cdot 10^2}
  = \frac{10^{-2}}{5184}
  = \frac{10^{-5}}{5,184} \notag
\end{align}
\setcounter{equation}{0}
\clearpage

\subsubsection{(b)}
$ln(1,3)$
\begin{align}
  P_2(x)
  &= f(x_0) + f'(x_0)(x-x_0) + \frac{f''(x_0)(x-x_0)^2} {2!} \\
  %%%%%%%%%%%%%%%%%%%%%%%%%%%%%%%%%%%%%%%%%%%%%%%%%%%%%%%%%%%%%%%%%%%%%%%%%%%%%%
  E_2(x)
  &= \frac{f'''(\bar{x}) \cdot (x-x_0)^3}{3!} \\
  %%%%%%%%%%%%%%%%%%%%%%%%%%%%%%%%%%%%%%%%%%%%%%%%%%%%%%%%%%%%%%%%%%%%%%%%%%%%%%
  f(x)
  &= ln(x) \\
  %%%%%%%%%%%%%%%%%%%%%%%%%%%%%%%%%%%%%%%%%%%%%%%%%%%%%%%%%%%%%%%%%%%%%%%%%%%%%%
  f'(x)
  &= \frac{1}{x}
  = x^{-1} \\
  %%%%%%%%%%%%%%%%%%%%%%%%%%%%%%%%%%%%%%%%%%%%%%%%%%%%%%%%%%%%%%%%%%%%%%%%%%%%%%
  f''(x)
  &= -x^{-2} \\
  %%%%%%%%%%%%%%%%%%%%%%%%%%%%%%%%%%%%%%%%%%%%%%%%%%%%%%%%%%%%%%%%%%%%%%%%%%%%%%
  f'''(x)
  &= 2x^{-3}
  = \frac{2}{x^3}
  && \text{O maior erro possível ocorre quando $\bar{x} = 1$} \\
  %%%%%%%%%%%%%%%%%%%%%%%%%%%%%%%%%%%%%%%%%%%%%%%%%%%%%%%%%%%%%%%%%%%%%%%%%%%%%%
  x_0
  &= 1 \\
  %%%%%%%%%%%%%%%%%%%%%%%%%%%%%%%%%%%%%%%%%%%%%%%%%%%%%%%%%%%%%%%%%%%%%%%%%%%%%%
  f(1)
  &= 0 \\
  %%%%%%%%%%%%%%%%%%%%%%%%%%%%%%%%%%%%%%%%%%%%%%%%%%%%%%%%%%%%%%%%%%%%%%%%%%%%%%
  f'(1)
  &= \frac{1}{1}
  = 1 \\
  %%%%%%%%%%%%%%%%%%%%%%%%%%%%%%%%%%%%%%%%%%%%%%%%%%%%%%%%%%%%%%%%%%%%%%%%%%%%%%
  f''(1)
  &= -1^{-2}
  = -\frac{1}{1^2}
  = -1 \\
  %%%%%%%%%%%%%%%%%%%%%%%%%%%%%%%%%%%%%%%%%%%%%%%%%%%%%%%%%%%%%%%%%%%%%%%%%%%%%%
  P_2(1,3)
  &= (1,3 - 1) -\frac{(1,3 - 1)^2}{2!}
  = 0,3 - \frac{\frac{3^2}{10^2}}{2} \\
  &= 0,3 -\frac{9}{200}
  = 0,255 \notag \\
  %%%%%%%%%%%%%%%%%%%%%%%%%%%%%%%%%%%%%%%%%%%%%%%%%%%%%%%%%%%%%%%%%%%%%%%%%%%%%%
  |E_2(1,3)|
  &= \left|\frac{\frac{2}{\bar{x}^3} \cdot (1,3 - 1)^3}{3!}\right|
  \leq \frac{\frac{2}{1^3} \cdot 0,3^3}{6}
  = \frac{2 \cdot 3^3}{10^3 \cdot 6} \\
  &= \frac{3^2}{10^3}
  = 3^2 \cdot 10^{-3} \notag
\end{align}
\setcounter{equation}{0}
\clearpage

\subsubsection{(c)}
$sin(0,1)$

\begin{align}
  P_2(x)
  &= f(x_0) + f'(x_0)(x-x_0) + \frac{f''(x_0)(x-x_0)^2} {2!} \\
  %%%%%%%%%%%%%%%%%%%%%%%%%%%%%%%%%%%%%%%%%%%%%%%%%%%%%%%%%%%%%%%%%%%%%%%%%%%%%%
  E_2(x)
  &= \frac{f'''(\bar{x}) \cdot (x-x_0)^3}{3!} \\
  %%%%%%%%%%%%%%%%%%%%%%%%%%%%%%%%%%%%%%%%%%%%%%%%%%%%%%%%%%%%%%%%%%%%%%%%%%%%%%
  f(x)
  &= sin(x) \\
  %%%%%%%%%%%%%%%%%%%%%%%%%%%%%%%%%%%%%%%%%%%%%%%%%%%%%%%%%%%%%%%%%%%%%%%%%%%%%%
  f'(x)
  &= cos(x) \\
  %%%%%%%%%%%%%%%%%%%%%%%%%%%%%%%%%%%%%%%%%%%%%%%%%%%%%%%%%%%%%%%%%%%%%%%%%%%%%%
  f''(x)
  &= -sin(x) \\
  %%%%%%%%%%%%%%%%%%%%%%%%%%%%%%%%%%%%%%%%%%%%%%%%%%%%%%%%%%%%%%%%%%%%%%%%%%%%%%
  f'''(x)
  &= -cos(x)
  && \text{O maior erro possível ocorre quando $\bar{x} = 0$} \\
  %%%%%%%%%%%%%%%%%%%%%%%%%%%%%%%%%%%%%%%%%%%%%%%%%%%%%%%%%%%%%%%%%%%%%%%%%%%%%%
  x_0
  &= 0 \\
  %%%%%%%%%%%%%%%%%%%%%%%%%%%%%%%%%%%%%%%%%%%%%%%%%%%%%%%%%%%%%%%%%%%%%%%%%%%%%%
  f(0)
  &= sin(0)
  = 0 \\
  %%%%%%%%%%%%%%%%%%%%%%%%%%%%%%%%%%%%%%%%%%%%%%%%%%%%%%%%%%%%%%%%%%%%%%%%%%%%%%
  f'(0)
  &= cos(0)
  = 1 \\
  %%%%%%%%%%%%%%%%%%%%%%%%%%%%%%%%%%%%%%%%%%%%%%%%%%%%%%%%%%%%%%%%%%%%%%%%%%%%%%
  f''(0)
  &= -sin(0)
  = 0 \\
  %%%%%%%%%%%%%%%%%%%%%%%%%%%%%%%%%%%%%%%%%%%%%%%%%%%%%%%%%%%%%%%%%%%%%%%%%%%%%%
  P_2(0,1)
  &= 0,1 - 0
  = 0,1 \\
  %%%%%%%%%%%%%%%%%%%%%%%%%%%%%%%%%%%%%%%%%%%%%%%%%%%%%%%%%%%%%%%%%%%%%%%%%%%%%%
  |E_2(0,1)|
  &= \left|\frac{-cos(0) \cdot (0,1 - 0)^3}{3!}\right|
  \leq \frac{1 \cdot (0,1)^3}{6} \\
  &= \frac{1}{6 \cdot 10^3}
  = \frac{10^{-3}}{6}
\end{align}
\setcounter{equation}{0}
\clearpage
\end{document}
