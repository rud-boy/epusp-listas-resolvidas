\documentclass{article}
\usepackage[brazil]{babel}
\usepackage{amsmath}
\usepackage{amsfonts}
\usepackage{hyperref}
\usepackage{fancyhdr}
\usepackage{pgfplots}
\usepackage[
  a4paper,
  includeheadfoot,
  left = 1cm,
  right = 1cm,
  top = 0.5cm,
  bottom = 0.5cm
]{geometry}

% Última versão do pacote pgfplots no momento.
\pgfplotsset{compat=1.18}

% Remove numeração de seções.
\setcounter{secnumdepth}{0}

% Limpar estilos para uso de fancyhdr
\pagestyle{empty}

\author{Rudah C. Amaral}
\title{Lista 1 - Cálculo Diferencial e Integral II}
\date{21 de agosto de 2021}

%%%%%%%%%%%%%%%%%%%%%%%%%%%%%%%%%%%% TÍTULO %%%%%%%%%%%%%%%%%%%%%%%%%%%%%%%%%%%%

\begin{document}
\maketitle
\clearpage

%%%%%%%%%%%%%%%%%%%%%%%%%%%%%%%%%%% SUMÁRIO %%%%%%%%%%%%%%%%%%%%%%%%%%%%%%%%%%%%

\tableofcontents
\label{TOC}
\clearpage

%%%%%%%%%%%%%%%%%%%%%%%%%%%%%%%% FIM DO SUMÁRIO %%%%%%%%%%%%%%%%%%%%%%%%%%%%%%%%

% Estilos de fancyhdr
\pagestyle{fancy}
\cfoot{\hyperref[TOC]{Retornar para o sumário}}
\setlength{\headheight}{16pt}
\rfoot{Página \thepage}

%%%%%%%%%%%%%%%%%%%%%% EXERCÍCIOS DE POLINÔMIOS DE TAYLOR %%%%%%%%%%%%%%%%%%%%%%

\section{1. Polinômio de Taylor}

%%%%%%%%%%%%%%%%%%%%%%%%%%%%%%%%% EXERCÍCIO 1 %%%%%%%%%%%%%%%%%%%%%%%%%%%%%%%%%%

\subsection{1.}
Utilizando o polinômio de Taylor de ordem 2, calcule um valor aproximado e
avalie o erro.

\subsubsection{(a)}
$\sqrt[3]{8,2}$
\begin{align}
  P_2(x)
  &= f(x_0) + f'(x_0)(x-x_0) + \frac{f''(x_0)(x-x_0)^2} {2!} \\
  %%%%%%%%%%%%%%%%%%%%%%%%%%%%%%%%%%%%%%%%%%%%%%%%%%%%%%%%%%%%%%%%%%%%%%%%%%%%%%
  E_2(x)
  &= \frac{f'''(\bar{x}) \cdot (x-x_0)^3}{3!} \\
  %%%%%%%%%%%%%%%%%%%%%%%%%%%%%%%%%%%%%%%%%%%%%%%%%%%%%%%%%%%%%%%%%%%%%%%%%%%%%%
  f(x)
  &= \sqrt[3]{x}
  = x^{\frac{1}{3}} \\
  %%%%%%%%%%%%%%%%%%%%%%%%%%%%%%%%%%%%%%%%%%%%%%%%%%%%%%%%%%%%%%%%%%%%%%%%%%%%%%
  f'(x)
  &= \frac{x^{-\frac{2}{3}}}{3}
  = \frac{\sqrt[3]{\frac{1}{x}^2}}{3} \\
  %%%%%%%%%%%%%%%%%%%%%%%%%%%%%%%%%%%%%%%%%%%%%%%%%%%%%%%%%%%%%%%%%%%%%%%%%%%%%%
  f''(x)
  &= \frac{1}{3} \cdot \frac{-2x^{-\frac{5}{3}}}{3}
  = -\frac{2}{9x^{\frac{5}{3}}} \\
  %%%%%%%%%%%%%%%%%%%%%%%%%%%%%%%%%%%%%%%%%%%%%%%%%%%%%%%%%%%%%%%%%%%%%%%%%%%%%%
  f'''(x)
  &= - \frac{2}{9} \cdot \frac{-5x^{-\frac{8}{3}}}{3}
  = \frac{10}{27x^{\frac{8}{3}}}
  && \text{O maior erro possível ocorre quando $\bar{x} = 8$} \\
  %%%%%%%%%%%%%%%%%%%%%%%%%%%%%%%%%%%%%%%%%%%%%%%%%%%%%%%%%%%%%%%%%%%%%%%%%%%%%%
  x_0
  &= 8 \\
  %%%%%%%%%%%%%%%%%%%%%%%%%%%%%%%%%%%%%%%%%%%%%%%%%%%%%%%%%%%%%%%%%%%%%%%%%%%%%%
  f(8)
  &= \sqrt[3]{8}
  = 2 \\
  %%%%%%%%%%%%%%%%%%%%%%%%%%%%%%%%%%%%%%%%%%%%%%%%%%%%%%%%%%%%%%%%%%%%%%%%%%%%%%
  f'(8)
  &= \frac{\sqrt[3]{\frac{1}{8}^2}}{3}
  = \frac{\frac{1}{4}}{3}
  = \frac{1}{12} \\
  %%%%%%%%%%%%%%%%%%%%%%%%%%%%%%%%%%%%%%%%%%%%%%%%%%%%%%%%%%%%%%%%%%%%%%%%%%%%%%
  f''(8)
  &= -\frac{2}{9 \cdot 8^{\frac{5}{3}}}
  = -\frac{2}{9 \cdot \sqrt[3]{8^5}}
  = -\frac{2}{9 \cdot 2^5}
  = -\frac{1}{144} \\
  %%%%%%%%%%%%%%%%%%%%%%%%%%%%%%%%%%%%%%%%%%%%%%%%%%%%%%%%%%%%%%%%%%%%%%%%%%%%%%
  P_2(8,2)
  &= 2 + \frac{8,2 - 8}{12} - \frac{\frac{1}{144}(8,2 - 8)^2}{2}
  = 2 + \frac{\frac{2}{10}}{12} - \frac{\frac{1}{144} \cdot \frac{2^2}{10^2}}{2} \\
  &= 2 + \frac{1}{10 \cdot 6} - \frac{1}{144} \cdot \frac{2}{10^2}
  = 2 + \frac{1}{60} - \frac{1}{7200} \notag \\
  %%%%%%%%%%%%%%%%%%%%%%%%%%%%%%%%%%%%%%%%%%%%%%%%%%%%%%%%%%%%%%%%%%%%%%%%%%%%%%
  |E_2(8,2)|
  &= \left|\frac{\frac{10}{27 \cdot 8^{\frac{8}{3}}} \cdot (8,2 - 8)^3}{3!}\right|
  \leq \frac{\frac{10}{27 \cdot \sqrt[3]{8^8}} \cdot (0,2)^3}{6} \\
  &= \frac{10}{27 \cdot 2^8} \cdot \frac{2^3}{10^3 \cdot 6}
  = \frac{1}{3^4 \cdot 2^6 \cdot 10^2}
  = \frac{10^{-2}}{5184}
  = \frac{10^{-5}}{5,184} \notag
\end{align}
\setcounter{equation}{0}
\clearpage

\subsubsection{(b)}
$ln(1,3)$
\begin{align}
  P_2(x)
  &= f(x_0) + f'(x_0)(x-x_0) + \frac{f''(x_0)(x-x_0)^2} {2!} \\
  %%%%%%%%%%%%%%%%%%%%%%%%%%%%%%%%%%%%%%%%%%%%%%%%%%%%%%%%%%%%%%%%%%%%%%%%%%%%%%
  E_2(x)
  &= \frac{f'''(\bar{x}) \cdot (x-x_0)^3}{3!} \\
  %%%%%%%%%%%%%%%%%%%%%%%%%%%%%%%%%%%%%%%%%%%%%%%%%%%%%%%%%%%%%%%%%%%%%%%%%%%%%%
  f(x)
  &= ln(x) \\
  %%%%%%%%%%%%%%%%%%%%%%%%%%%%%%%%%%%%%%%%%%%%%%%%%%%%%%%%%%%%%%%%%%%%%%%%%%%%%%
  f'(x)
  &= \frac{1}{x}
  = x^{-1} \\
  %%%%%%%%%%%%%%%%%%%%%%%%%%%%%%%%%%%%%%%%%%%%%%%%%%%%%%%%%%%%%%%%%%%%%%%%%%%%%%
  f''(x)
  &= -x^{-2} \\
  %%%%%%%%%%%%%%%%%%%%%%%%%%%%%%%%%%%%%%%%%%%%%%%%%%%%%%%%%%%%%%%%%%%%%%%%%%%%%%
  f'''(x)
  &= 2x^{-3}
  = \frac{2}{x^3}
  && \text{O maior erro possível ocorre quando $\bar{x} = 1$} \\
  %%%%%%%%%%%%%%%%%%%%%%%%%%%%%%%%%%%%%%%%%%%%%%%%%%%%%%%%%%%%%%%%%%%%%%%%%%%%%%
  x_0
  &= 1 \\
  %%%%%%%%%%%%%%%%%%%%%%%%%%%%%%%%%%%%%%%%%%%%%%%%%%%%%%%%%%%%%%%%%%%%%%%%%%%%%%
  f(1)
  &= 0 \\
  %%%%%%%%%%%%%%%%%%%%%%%%%%%%%%%%%%%%%%%%%%%%%%%%%%%%%%%%%%%%%%%%%%%%%%%%%%%%%%
  f'(1)
  &= \frac{1}{1}
  = 1 \\
  %%%%%%%%%%%%%%%%%%%%%%%%%%%%%%%%%%%%%%%%%%%%%%%%%%%%%%%%%%%%%%%%%%%%%%%%%%%%%%
  f''(1)
  &= -1^{-2}
  = -\frac{1}{1^2}
  = -1 \\
  %%%%%%%%%%%%%%%%%%%%%%%%%%%%%%%%%%%%%%%%%%%%%%%%%%%%%%%%%%%%%%%%%%%%%%%%%%%%%%
  P_2(1,3)
  &= (1,3 - 1) -\frac{(1,3 - 1)^2}{2!}
  = 0,3 - \frac{\frac{3^2}{10^2}}{2} \\
  &= 0,3 -\frac{9}{200}
  = 0,255 \notag \\
  %%%%%%%%%%%%%%%%%%%%%%%%%%%%%%%%%%%%%%%%%%%%%%%%%%%%%%%%%%%%%%%%%%%%%%%%%%%%%%
  |E_2(1,3)|
  &= \left|\frac{\frac{2}{\bar{x}^3} \cdot (1,3 - 1)^3}{3!}\right|
  \leq \frac{\frac{2}{1^3} \cdot 0,3^3}{6}
  = \frac{2 \cdot 3^3}{10^3 \cdot 6} \\
  &= \frac{3^2}{10^3}
  = 3^2 \cdot 10^{-3} \notag
\end{align}
\setcounter{equation}{0}
\clearpage

\subsubsection{(c)}
$sin(0,1)$
\begin{align}
  P_2(x)
  &= f(x_0) + f'(x_0)(x-x_0) + \frac{f''(x_0)(x-x_0)^2} {2!} \\
  %%%%%%%%%%%%%%%%%%%%%%%%%%%%%%%%%%%%%%%%%%%%%%%%%%%%%%%%%%%%%%%%%%%%%%%%%%%%%%
  E_2(x)
  &= \frac{f'''(\bar{x}) \cdot (x-x_0)^3}{3!} \\
  %%%%%%%%%%%%%%%%%%%%%%%%%%%%%%%%%%%%%%%%%%%%%%%%%%%%%%%%%%%%%%%%%%%%%%%%%%%%%%
  f(x)
  &= sin(x) \\
  %%%%%%%%%%%%%%%%%%%%%%%%%%%%%%%%%%%%%%%%%%%%%%%%%%%%%%%%%%%%%%%%%%%%%%%%%%%%%%
  f'(x)
  &= cos(x) \\
  %%%%%%%%%%%%%%%%%%%%%%%%%%%%%%%%%%%%%%%%%%%%%%%%%%%%%%%%%%%%%%%%%%%%%%%%%%%%%%
  f''(x)
  &= -sin(x) \\
  %%%%%%%%%%%%%%%%%%%%%%%%%%%%%%%%%%%%%%%%%%%%%%%%%%%%%%%%%%%%%%%%%%%%%%%%%%%%%%
  f'''(x)
  &= -cos(x)
  && \text{O maior erro possível ocorre quando $\bar{x} = 0$} \\
  %%%%%%%%%%%%%%%%%%%%%%%%%%%%%%%%%%%%%%%%%%%%%%%%%%%%%%%%%%%%%%%%%%%%%%%%%%%%%%
  x_0
  &= 0 \\
  %%%%%%%%%%%%%%%%%%%%%%%%%%%%%%%%%%%%%%%%%%%%%%%%%%%%%%%%%%%%%%%%%%%%%%%%%%%%%%
  f(0)
  &= sin(0)
  = 0 \\
  %%%%%%%%%%%%%%%%%%%%%%%%%%%%%%%%%%%%%%%%%%%%%%%%%%%%%%%%%%%%%%%%%%%%%%%%%%%%%%
  f'(0)
  &= cos(0)
  = 1 \\
  %%%%%%%%%%%%%%%%%%%%%%%%%%%%%%%%%%%%%%%%%%%%%%%%%%%%%%%%%%%%%%%%%%%%%%%%%%%%%%
  f''(0)
  &= -sin(0)
  = 0 \\
  %%%%%%%%%%%%%%%%%%%%%%%%%%%%%%%%%%%%%%%%%%%%%%%%%%%%%%%%%%%%%%%%%%%%%%%%%%%%%%
  P_2(0,1)
  &= 0,1 - 0
  = 0,1 \\
  %%%%%%%%%%%%%%%%%%%%%%%%%%%%%%%%%%%%%%%%%%%%%%%%%%%%%%%%%%%%%%%%%%%%%%%%%%%%%%
  |E_2(0,1)|
  &= \left|\frac{-cos(0) \cdot (0,1 - 0)^3}{3!}\right|
  \leq \frac{1 \cdot (0,1)^3}{6} \\
  &= \frac{1}{6 \cdot 10^3}
  = \frac{10^{-3}}{6}
\end{align}
\setcounter{equation}{0}
\clearpage

%%%%%%%%%%%%%%%%%%%%%%%%%%%%%%%%% EXERCÍCIO 2 %%%%%%%%%%%%%%%%%%%%%%%%%%%%%%%%%%

\subsection{2.}
Mostre que:

\subsubsection{(a)}
$|sin \, x - x| \leq \frac{|x^3|}{3!}, \forall x \in \mathbb{R}$.
\begin{align}
  P_2(x)
  &= f(x_0) + f'(x_0)(x - x_0) + \frac{f''(x)(x - x_0)^2}{2!} \\
  %%%%%%%%%%%%%%%%%%%%%%%%%%%%%%%%%%%%%%%%%%%%%%%%%%%%%%%%%%%%%%%%%%%%%%%%%%%%%%
  f(x)
  &= sin(x) \\
  %%%%%%%%%%%%%%%%%%%%%%%%%%%%%%%%%%%%%%%%%%%%%%%%%%%%%%%%%%%%%%%%%%%%%%%%%%%%%%
  f'(x)
  &= cos(x) \\
  %%%%%%%%%%%%%%%%%%%%%%%%%%%%%%%%%%%%%%%%%%%%%%%%%%%%%%%%%%%%%%%%%%%%%%%%%%%%%%
  f''(x)
  &= -sin(x) \\
  %%%%%%%%%%%%%%%%%%%%%%%%%%%%%%%%%%%%%%%%%%%%%%%%%%%%%%%%%%%%%%%%%%%%%%%%%%%%%%
  x_0
  &= 0 \\
  %%%%%%%%%%%%%%%%%%%%%%%%%%%%%%%%%%%%%%%%%%%%%%%%%%%%%%%%%%%%%%%%%%%%%%%%%%%%%%
  f(x_0)
  &= sin(0)
  = 0 \\
  %%%%%%%%%%%%%%%%%%%%%%%%%%%%%%%%%%%%%%%%%%%%%%%%%%%%%%%%%%%%%%%%%%%%%%%%%%%%%%
  f'(x_0)
  &= cos(0)
  = 1 \\
  %%%%%%%%%%%%%%%%%%%%%%%%%%%%%%%%%%%%%%%%%%%%%%%%%%%%%%%%%%%%%%%%%%%%%%%%%%%%%%
  f''(x_0)
  &= -sin(0)
  = 0 \\
  %%%%%%%%%%%%%%%%%%%%%%%%%%%%%%%%%%%%%%%%%%%%%%%%%%%%%%%%%%%%%%%%%%%%%%%%%%%%%%
  P_2(x)
  &= 0 + 1(x - 0) + \frac{0(x - 0)^2}{2!}
  = x
  && \text{Logo:} \\
  %%%%%%%%%%%%%%%%%%%%%%%%%%%%%%%%%%%%%%%%%%%%%%%%%%%%%%%%%%%%%%%%%%%%%%%%%%%%%%
  |sin \, x - x|
  &= |f(x) - P_2(x)|
  = |E_2(x)| \\
  %%%%%%%%%%%%%%%%%%%%%%%%%%%%%%%%%%%%%%%%%%%%%%%%%%%%%%%%%%%%%%%%%%%%%%%%%%%%%%
  E_2(x)
  &= \frac{f'''(\bar{x}) (x - x_0)^3}{3!} \\
  %%%%%%%%%%%%%%%%%%%%%%%%%%%%%%%%%%%%%%%%%%%%%%%%%%%%%%%%%%%%%%%%%%%%%%%%%%%%%%
  f'''(x)
  &= -cos(x)
  && \text{O valor máximo absoluto dessa função é 1, logo:} \\
  %%%%%%%%%%%%%%%%%%%%%%%%%%%%%%%%%%%%%%%%%%%%%%%%%%%%%%%%%%%%%%%%%%%%%%%%%%%%%%
  |E_2(x)|
  &= \left|\frac{f'''(\bar{x}) (x - x_0)^3}{3!}\right|
  \leq \frac{1\left|(x - 0)^3\right|}{3!}
  = \frac{|x^3|}{3!}
  && \text{Portanto:} \\
  %%%%%%%%%%%%%%%%%%%%%%%%%%%%%%%%%%%%%%%%%%%%%%%%%%%%%%%%%%%%%%%%%%%%%%%%%%%%%%
  |sin \, x - x|
  &= |E_2(x)|
  \leq \frac{|x^3|}{3!}
\end{align}
\setcounter{equation}{0}
\clearpage

\subsubsection{(b)}
$0
\leq e^x - \left(1 + x + \frac{x^2}{2}\right)
< \frac{x^3}{2}, \forall x \in [0, 1]$.
\begin{align}
  P_2(x)
  &= f(x_0) + f'(x_0)(x - x_0) + \frac{f''(x)(x - x_0)^2}{2!} \\
  %%%%%%%%%%%%%%%%%%%%%%%%%%%%%%%%%%%%%%%%%%%%%%%%%%%%%%%%%%%%%%%%%%%%%%%%%%%%%%
  f(x)
  &= e^x \\
  %%%%%%%%%%%%%%%%%%%%%%%%%%%%%%%%%%%%%%%%%%%%%%%%%%%%%%%%%%%%%%%%%%%%%%%%%%%%%%
  f'(x)
  &= e^x \\
  %%%%%%%%%%%%%%%%%%%%%%%%%%%%%%%%%%%%%%%%%%%%%%%%%%%%%%%%%%%%%%%%%%%%%%%%%%%%%%
  f''(x)
  &= e^x \\
  %%%%%%%%%%%%%%%%%%%%%%%%%%%%%%%%%%%%%%%%%%%%%%%%%%%%%%%%%%%%%%%%%%%%%%%%%%%%%%
  x_0
  &= 0 \\
  %%%%%%%%%%%%%%%%%%%%%%%%%%%%%%%%%%%%%%%%%%%%%%%%%%%%%%%%%%%%%%%%%%%%%%%%%%%%%%
  f(x_0)
  &= e^0
  = 1 \\
  %%%%%%%%%%%%%%%%%%%%%%%%%%%%%%%%%%%%%%%%%%%%%%%%%%%%%%%%%%%%%%%%%%%%%%%%%%%%%%
  f'(x_0)
  &= e^0
  = 1 \\
  %%%%%%%%%%%%%%%%%%%%%%%%%%%%%%%%%%%%%%%%%%%%%%%%%%%%%%%%%%%%%%%%%%%%%%%%%%%%%%
  f''(x_0)
  &= e^0
  = 1 \\
  %%%%%%%%%%%%%%%%%%%%%%%%%%%%%%%%%%%%%%%%%%%%%%%%%%%%%%%%%%%%%%%%%%%%%%%%%%%%%%
  P_2(x)
  &= 1 + 1(x - 0) + \frac{1(x - 0)^2}{2!} \\
  &= 1 + x + \frac{x^2}{2}
  && \text{Logo:} \notag \\
  %%%%%%%%%%%%%%%%%%%%%%%%%%%%%%%%%%%%%%%%%%%%%%%%%%%%%%%%%%%%%%%%%%%%%%%%%%%%%%
  e^x - \left(1 + x + \frac{x^2}{2}\right)
  &= f(x) - P_2(x)
  = E_2(x) \\
  %%%%%%%%%%%%%%%%%%%%%%%%%%%%%%%%%%%%%%%%%%%%%%%%%%%%%%%%%%%%%%%%%%%%%%%%%%%%%%
  E_2
  &= \frac{f'''(\bar{x}) (x - x_0)^3}{3!} \\
  %%%%%%%%%%%%%%%%%%%%%%%%%%%%%%%%%%%%%%%%%%%%%%%%%%%%%%%%%%%%%%%%%%%%%%%%%%%%%%
  f'''(x)
  &= e^x
  && \text{O maior erro possível ocorre quando $\bar{x} = 1$} \\
  & && \text{O menor erro possível ocorre quando $\bar{x} = 0$} \notag \\
  %%%%%%%%%%%%%%%%%%%%%%%%%%%%%%%%%%%%%%%%%%%%%%%%%%%%%%%%%%%%%%%%%%%%%%%%%%%%%%
  E_2(x)
  &= \frac{f'''(\bar{x}) (x - x_0)^3}{3!}
  \leq \frac{e(x - 0)^3}{3!}
  = \frac{ex^3}{6} \\
  %%%%%%%%%%%%%%%%%%%%%%%%%%%%%%%%%%%%%%%%%%%%%%%%%%%%%%%%%%%%%%%%%%%%%%%%%%%%%%
  E_2(x)
  &= \frac{f'''(\bar{x}) (x - x_0)^3}{3!}
  \geq \frac{1(x - 0)^3}{3!}
  = \frac{x^3}{6}
  && \text{Portanto:} \\
  %%%%%%%%%%%%%%%%%%%%%%%%%%%%%%%%%%%%%%%%%%%%%%%%%%%%%%%%%%%%%%%%%%%%%%%%%%%%%%
  0 \leq \frac{x^3}{6}
  &\leq E_2 \leq \frac{ex^3}{6}
  < \frac{3x^3}{6} = \frac{x^3}{2}, \forall x \in [0,1]
\end{align}
\setcounter{equation}{0}
\clearpage

%%%%%%%%%%%%%%%%%%%%%%%%%%%%%%%%% EXERCÍCIO 3 %%%%%%%%%%%%%%%%%%%%%%%%%%%%%%%%%%

\subsection{3.}
Determine o polinômio de Taylor de order 5 da função $f(x) = \sqrt[3]{x}$ em
torno de $x_0 = 1$.
\begin{align}
  P_5(x)
  &= f(x_0)
  + f'(x_0)(x - x_0)
  + \frac{f''(x_0)(x - x_0)^2}{2}
  + \frac{f'''(x_0)(x - x_0)^3}{3!}
  + \frac{f^{(4)}(x_0)(x - x_0)^4}{4!}
  + \frac{f^{(5)}(x_0)(x - x_0)^5}{5!} \\
  %%%%%%%%%%%%%%%%%%%%%%%%%%%%%%%%%%%%%%%%%%%%%%%%%%%%%%%%%%%%%%%%%%%%%%%%%%%%%%
  f(x)
  &= \sqrt[3]{x}
  = x^{\frac{1}{3}}\\
  %%%%%%%%%%%%%%%%%%%%%%%%%%%%%%%%%%%%%%%%%%%%%%%%%%%%%%%%%%%%%%%%%%%%%%%%%%%%%%
  f'(x)
  &= \frac{x^{-\frac{2}{3}}}{3}
  = \frac{1}{3 \sqrt[3]{x^2}} \\
  %%%%%%%%%%%%%%%%%%%%%%%%%%%%%%%%%%%%%%%%%%%%%%%%%%%%%%%%%%%%%%%%%%%%%%%%%%%%%%
  f''(x)
  &= -\frac{2 \cdot x^{-\frac{5}{3}}}{3^2}
  = -\frac{2}{3^2 \sqrt[3]{x^5}} \\
  %%%%%%%%%%%%%%%%%%%%%%%%%%%%%%%%%%%%%%%%%%%%%%%%%%%%%%%%%%%%%%%%%%%%%%%%%%%%%%
  f'''(x)
  &= \frac{2 \cdot 5 \cdot x^{-\frac{8}{3}}}{3^3}
  = \frac{10}{3^3 \sqrt[3]{x^8}} \\
  %%%%%%%%%%%%%%%%%%%%%%%%%%%%%%%%%%%%%%%%%%%%%%%%%%%%%%%%%%%%%%%%%%%%%%%%%%%%%%
  f^{(4)}(x)
  &= -\frac{2 \cdot 5 \cdot 8 \cdot x^{-\frac{11}{3}}}{3^4}
  = -\frac{80}{3^4 \sqrt[3]{x^{11}}} \\
  %%%%%%%%%%%%%%%%%%%%%%%%%%%%%%%%%%%%%%%%%%%%%%%%%%%%%%%%%%%%%%%%%%%%%%%%%%%%%%
  f^{(5)}(x)
  &= \frac{2 \cdot 5 \cdot 8 \cdot 11 \cdot x^{-\frac{14}{3}}}{3^5}
  = \frac{880}{3^5 \sqrt[3]{x^{14}}} \\
  %%%%%%%%%%%%%%%%%%%%%%%%%%%%%%%%%%%%%%%%%%%%%%%%%%%%%%%%%%%%%%%%%%%%%%%%%%%%%%
  P_5(x)
  &= 1
  + \frac{1}{3}(x - 1)
  - \frac{\frac{2}{3^2}(x - 1)^2}{2}
  + \frac{\frac{10}{3^3}(x - 1)^3}{6}
  - \frac{\frac{80}{3^4}(x - 1)^4}{24}
  + \frac{\frac{880}{3^5}(x - 1)^5}{120} \\
  &= 1
  + \frac{x - 1}{3}
  - \frac{(x - 1)^2}{3^2}
  + \frac{5(x - 1)^3}{3^4}
  - \frac{10(x - 1)^4}{3^5}
  + \frac{22(x - 1)^5}{3^6} \notag
\end{align}
\setcounter{equation}{0}
\clearpage

%%%%%%%%%%%%%%%%%%%%%%%%%%%%%%%%% EXERCÍCIO 4 %%%%%%%%%%%%%%%%%%%%%%%%%%%%%%%%%%

\subsection{4.}
Determine $P3(x)$, o polinômio de Taylor de ordem 3 da função
$f(x) = \sqrt[5]{x}$ em torno de $x_0$ e dê a fórmula para o erro
$E(x) = f(x) - P_3(x)$.  Use este polinômio com um $x_0$ conveniente para
avaliar $\sqrt[5]{34}$ com erro inferior a $5^{-2} \cdot 2^{-15}$.
\begin{align}
  f(x)
  &= \sqrt[5]{x}
  = x^{\frac{1}{5}} \\
  %%%%%%%%%%%%%%%%%%%%%%%%%%%%%%%%%%%%%%%%%%%%%%%%%%%%%%%%%%%%%%%%%%%%%%%%%%%%%%
  f'(x)
  &= \frac{x^{-\frac{4}{5}}}{5}
  = \frac{\frac{1}{\sqrt[5]{x^4}}}{5}
  = \frac{1}{5\sqrt[5]{x^4}} \\
  %%%%%%%%%%%%%%%%%%%%%%%%%%%%%%%%%%%%%%%%%%%%%%%%%%%%%%%%%%%%%%%%%%%%%%%%%%%%%%
  f''(x)
  &= -\frac{4x^{-\frac{9}{5}}}{5^2}
  = -\frac{4 \cdot \frac{1}{\sqrt[5]{x^9}}}{5^2}
  = -\frac{4}{5^2 \sqrt[5]{x^9}} \\
  %%%%%%%%%%%%%%%%%%%%%%%%%%%%%%%%%%%%%%%%%%%%%%%%%%%%%%%%%%%%%%%%%%%%%%%%%%%%%%
  f'''(x)
  &= \frac{4 \cdot 9x^{-\frac{14}{5}}}{5^3}
  = \frac{4 \cdot 9 \cdot \frac{1}{\sqrt[5]{x^{14}}}}{5^3}
  = \frac{4 \cdot 9}{5^3\sqrt[5]{x^{14}}} \\
  %%%%%%%%%%%%%%%%%%%%%%%%%%%%%%%%%%%%%%%%%%%%%%%%%%%%%%%%%%%%%%%%%%%%%%%%%%%%%%
  f^{(4)}(x)
  &= -\frac{4 \cdot 9 \cdot 14x^{-\frac{19}{5}}}{5^4}
  = -\frac{4 \cdot 9 \cdot 14 \cdot \frac{1}{\sqrt[5]{x^{19}}}}{5^4}
  = -\frac{4 \cdot 9 \cdot 14}{5^4 \sqrt[5]{x^{19}}}
  && \text{Quanto menor o $\bar{x}$, maior o erro} \\
  %%%%%%%%%%%%%%%%%%%%%%%%%%%%%%%%%%%%%%%%%%%%%%%%%%%%%%%%%%%%%%%%%%%%%%%%%%%%%%
  P_3(x)
  &= f(x_0) + f'(x_0)(x - x_0) + \frac{f''(x_0)(x - x_0)^2}{2} +
  \frac{f'''(x_0)(x - x_0)^3}{3!} \\
  &= \sqrt[5]{x_0} +
  \frac{x - x_0}{5\sqrt[5]{x_0^4}} -
  \frac{4(x - x_0)^2}{5^2 \sqrt[5]{x_0^9} \cdot 2} +
  \frac{4 \cdot 9(x - x_0)^3}{5^3 \sqrt[5]{x_0^{14}} \cdot 3!} \notag \\
  %%%%%%%%%%%%%%%%%%%%%%%%%%%%%%%%%%%%%%%%%%%%%%%%%%%%%%%%%%%%%%%%%%%%%%%%%%%%%%
  E_3(x)
  &= \frac{f^{(4)}(\bar{x})(x - x_0)^4}{4!}
  = \frac{-\frac{4 \cdot 9 \cdot 14}{5^4 \sqrt[5]{\bar{x}^{19}}}(x - x_0)^4}{4!}
  = -\frac{4 \cdot 9 \cdot 14(x - x_0)^4}{5^4\sqrt[5]{\bar{x}^{19}} \cdot 4!} \\
  %%%%%%%%%%%%%%%%%%%%%%%%%%%%%%%%%%%%%%%%%%%%%%%%%%%%%%%%%%%%%%%%%%%%%%%%%%%%%%
  x_0
  &= 32 \\
  %%%%%%%%%%%%%%%%%%%%%%%%%%%%%%%%%%%%%%%%%%%%%%%%%%%%%%%%%%%%%%%%%%%%%%%%%%%%%%
  P_3(34)
  &= \sqrt[5]{32} +
  \frac{2}{5 \sqrt[5]{32^4}} -
  \frac{4(2)^2}{5^2 \sqrt[5]{32^9} \cdot 2} +
  \frac{4 \cdot 9(2)^3}{5^3 \sqrt[5]{32^{14}} \cdot 6} \\
  &= 2 + \frac{1}{5 \cdot 2^3}
  -\frac{1}{5^2 \cdot 2^6}
  + \frac{3}{5^3 \cdot 2^{10}} \notag \\
  &= 2 + \frac{1}{40}
  -\frac{1}{1600}
  + \frac{3}{128000} \notag \\
  %%%%%%%%%%%%%%%%%%%%%%%%%%%%%%%%%%%%%%%%%%%%%%%%%%%%%%%%%%%%%%%%%%%%%%%%%%%%%%
  |E_3(34)|
  &= \left|-\frac{4 \cdot 9 \cdot 14(2)^4}
                 {5^4 \sqrt[5]{\bar{x}^{19}} \cdot 4!}\right|
  \leq \frac{2^7 \cdot 3^2 \cdot 7}{5^4\sqrt[5]{32^{19}} \cdot 24}
  = \frac{2^7 \cdot 3^2 \cdot 7}{5^4 \cdot 2^{19} \cdot 2^3 \cdot 3} \\
  &= \frac{3 \cdot 7}{5^4 \cdot 2^{15}}
  \leq \frac{5^2}{5^4 \cdot 2^{15}}
  = 5^{-2} \cdot 2^{-15}
\end{align}
\setcounter{equation}{0}
\clearpage

%%%%%%%%%%%%%%%%%%%%%%%%%%%%%%%%% EXERCÍCIO 5 %%%%%%%%%%%%%%%%%%%%%%%%%%%%%%%%%%

\subsection{5.}

\subsubsection{(a)}
Seja $n > 0$ um inteiro ímpar. Mostre que
\[
  \left| sin \, x
  - \left(x - \frac{x^3}{3!} + \frac{x^5}{5!} + \dots
  + \frac{(-1)\frac{n-1}{2} x^n}{n!}\right)\right|
  \leq \frac{|x^{n+2}|}{(n+2)!}, \forall x \in \mathbb{R}.
\]
\begin{align}
  P_n(x)
  &= \sum_0^n \frac{f^{(n)}(x_0)(x-x_0)^n}{n!} \\
  %%%%%%%%%%%%%%%%%%%%%%%%%%%%%%%%%%%%%%%%%%%%%%%%%%%%%%%%%%%%%%%%%%%%%%%%%%%%%%
  f(x)
  &= f^{(0 + 4n)}(x)
  = sin(x) \\
  %%%%%%%%%%%%%%%%%%%%%%%%%%%%%%%%%%%%%%%%%%%%%%%%%%%%%%%%%%%%%%%%%%%%%%%%%%%%%%
  f'(x)
  &= f^{(1 + 4n)}(x)
  = cos(x) \\
  %%%%%%%%%%%%%%%%%%%%%%%%%%%%%%%%%%%%%%%%%%%%%%%%%%%%%%%%%%%%%%%%%%%%%%%%%%%%%%
  f''(x)
  &= f^{(2 + 4n)}(x)
  = -sin(x) \\
  %%%%%%%%%%%%%%%%%%%%%%%%%%%%%%%%%%%%%%%%%%%%%%%%%%%%%%%%%%%%%%%%%%%%%%%%%%%%%%
  f'''(x)
  &= f^{(3 + 4n)}(x)
  = -cos(x) \\
  %%%%%%%%%%%%%%%%%%%%%%%%%%%%%%%%%%%%%%%%%%%%%%%%%%%%%%%%%%%%%%%%%%%%%%%%%%%%%%
  x_0
  &= 0
  && \text{Para eliminar as parcelas com derivadas pares.} \\
  %%%%%%%%%%%%%%%%%%%%%%%%%%%%%%%%%%%%%%%%%%%%%%%%%%%%%%%%%%%%%%%%%%%%%%%%%%%%%%
  f(x_0)
  &= f^{(0 + 4n)}(x_0)
  = sin(0)
  = 0 \\
  %%%%%%%%%%%%%%%%%%%%%%%%%%%%%%%%%%%%%%%%%%%%%%%%%%%%%%%%%%%%%%%%%%%%%%%%%%%%%%
  f'(x_0)
  &= f^{(1 + 4n)}(x_0)
  = cos(0)
  = 1 \\
  %%%%%%%%%%%%%%%%%%%%%%%%%%%%%%%%%%%%%%%%%%%%%%%%%%%%%%%%%%%%%%%%%%%%%%%%%%%%%%
  f''(x_0)
  &= f^{(2 + 4n)}(x_0)
  = -sin(0)
  = 0 \\
  %%%%%%%%%%%%%%%%%%%%%%%%%%%%%%%%%%%%%%%%%%%%%%%%%%%%%%%%%%%%%%%%%%%%%%%%%%%%%%
  f'''(x_0)
  &= f^{(3 + 4n)}(x_0)
  = -cos(0)
  = -1 \\
  %%%%%%%%%%%%%%%%%%%%%%%%%%%%%%%%%%%%%%%%%%%%%%%%%%%%%%%%%%%%%%%%%%%%%%%%%%%%%%
  P_n(x)
  &= \frac{x^1}{1!}
  - \frac{x^3}{3!}
  + \frac{x^5}{5!} + \dots + \frac{(-1)^{\frac{n-1}{2}}x^n}{n!} \\
  %%%%%%%%%%%%%%%%%%%%%%%%%%%%%%%%%%%%%%%%%%%%%%%%%%%%%%%%%%%%%%%%%%%%%%%%%%%%%%
  |f(x) - P_n(x)|
  &= |f(x) - P_{n+1}(x)|
  = |E_{n+1}(x)|
  && \text{$P_n(x) = P_{n+1}(x)$ quando $n > 0$ e é ímpar} \\
  %%%%%%%%%%%%%%%%%%%%%%%%%%%%%%%%%%%%%%%%%%%%%%%%%%%%%%%%%%%%%%%%%%%%%%%%%%%%%%
  |E_{n + 1}(x)|
  &= \left|\frac{f^{(n+2)}(\bar{x}) (x-x_0)^{n+2}}{(n+2)!}\right|
  && \text{O maior valor absoluto possível para $f^{(n+2)}(\bar{x})$ é 1} \\
  & \leq \frac{|x^{n+2}|}{(n+2)!} \notag
\end{align}
\setcounter{equation}{0}
\clearpage

\subsubsection{(b)}
Avalie $sin \, 1$ com erro inferior a $10^{-5}$.
\begin{align}
  |E_{n+1}(1)|
  &< 10^{-5} \\
  %%%%%%%%%%%%%%%%%%%%%%%%%%%%%%%%%%%%%%%%%%%%%%%%%%%%%%%%%%%%%%%%%%%%%%%%%%%%%%
  \frac{|1^{n+2}|}{(n+2)!}
  &< 10^{-5} \notag \\
  %%%%%%%%%%%%%%%%%%%%%%%%%%%%%%%%%%%%%%%%%%%%%%%%%%%%%%%%%%%%%%%%%%%%%%%%%%%%%%
  \frac{1}{10^{-5}}
  &< (n + 2)! \notag \\
  %%%%%%%%%%%%%%%%%%%%%%%%%%%%%%%%%%%%%%%%%%%%%%%%%%%%%%%%%%%%%%%%%%%%%%%%%%%%%%
  (n + 2)!
  &> 10^5
  && \text{Procuramos o menor n possível que satisfaça essa equação} \notag \\
%%%%%%%%%%%%%%%%%%%%%%%%%%%%%%%%%%%%%%%%%%%%%%%%%%%%%%%%%%%%%%%%%%%%%%%%%%%%%%%%
  \notag \\
  &
  \begin{tabular}{| c | c | r |}
    \hline
    \textbf{n} & \textbf{n+2} & \textbf{(n+2)!}  \\ \hline
            1  &         3    &            6     \\ \hline
            3  &         5    &          120     \\ \hline
            5  &         7    &         5040     \\ \hline
            7  &         9    &       362880     \\ \hline
  \end{tabular} \notag \\
  \notag \\
%%%%%%%%%%%%%%%%%%%%%%%%%%%%%%%%%%%%%%%%%%%%%%%%%%%%%%%%%%%%%%%%%%%%%%%%%%%%%%%%
  P_7(1)
  &= \frac{1^1}{1!}
  - \frac{1^3}{3!}
  + \frac{1^5}{5!}
  - \frac{1^7}{7!} \\
  &= 1
  - \frac{1}{6}
  + \frac{1}{120}
  - \frac{1}{5040} \notag \\
  &= 0,841468253 \ldots \notag
\end{align}
\setcounter{equation}{0}
\clearpage

%%%%%%%%%%%%%%%%%%%%%%%%%%%%%%%%% EXERCÍCIO 6 %%%%%%%%%%%%%%%%%%%%%%%%%%%%%%%%%%

\subsection{6.}

\subsubsection{(a)}
Determine o Polinômio de Taylor de ordem $n$ na função $f(x) = e^x$ em torno de
$x_0 = 0$.
\begin{align}
  P_n(x)
  &= f(x_0) + \sum_{i = 1}^n \frac{f^{(i)}(x_0)(x-x_0)^i}{i!} \\
  %%%%%%%%%%%%%%%%%%%%%%%%%%%%%%%%%%%%%%%%%%%%%%%%%%%%%%%%%%%%%%%%%%%%%%%%%%%%%%
  f^{n}(x_0)
  &= f^{n}(0)
  = e^0
  = 1 \\
  %%%%%%%%%%%%%%%%%%%%%%%%%%%%%%%%%%%%%%%%%%%%%%%%%%%%%%%%%%%%%%%%%%%%%%%%%%%%%%
  P_n(x)
  &= 1 + x + \frac{x^2}{2!} + \frac{x^3}{3!} + \dots + \frac{x^n}{n!}
\end{align}
\setcounter{equation}{0}
\clearpage

\subsubsection{(b)}
Avalie $e$ com erro, em módulo, inferior a $10^{-5}$.
\begin{align}
  |E_n(x)|
  = \left| \frac{f^{(n+1)}(\bar{x})(x-x_0)^{n+1}}{(n + 1)!} \right|
  &\leq 10^{-5}
  && 0 < \bar{x} < 1 \\
  %%%%%%%%%%%%%%%%%%%%%%%%%%%%%%%%%%%%%%%%%%%%%%%%%%%%%%%%%%%%%%%%%%%%%%%%%%%%%%
  \frac{e^{\bar{x}} \cdot |x^{n+1}|}{(n + 1)!}
  \leq \frac{3 \cdot |1^{n+1}|}{(n + 1)!}
  &\leq 10^{-5}
  && \text{O maior erro possível ocorre quando $\bar{x} = 1$} \notag \\
  %%%%%%%%%%%%%%%%%%%%%%%%%%%%%%%%%%%%%%%%%%%%%%%%%%%%%%%%%%%%%%%%%%%%%%%%%%%%%%
  3 \cdot 10^5
  &\leq (n + 1)!
  && \text{Procuramos o menor $n$ possível que satisfaça essa equação} \notag \\
  %%%%%%%%%%%%%%%%%%%%%%%%%%%%%%%%%%%%%%%%%%%%%%%%%%%%%%%%%%%%%%%%%%%%%%%%%%%%%%
  \notag \\
  &
  \begin{tabular}{| c | c | r |}
    \hline
    \textbf{n} & \textbf{n+1} & \textbf{(n+1)!}  \\ \hline
            1  &         2    &            2     \\ \hline
            2  &         3    &            6     \\ \hline
            3  &         4    &           24     \\ \hline
            4  &         5    &          120     \\ \hline
            5  &         6    &          720     \\ \hline
            6  &         7    &         5040     \\ \hline
            7  &         8    &        40320     \\ \hline
            8  &         9    &       362880     \\ \hline
  \end{tabular} \notag \\
  \notag \\
  %%%%%%%%%%%%%%%%%%%%%%%%%%%%%%%%%%%%%%%%%%%%%%%%%%%%%%%%%%%%%%%%%%%%%%%%%%%%%%
  P_8(1)
  &= \sum_{i = 0}^8 \frac{1}{i!}
  = 2,71827877 \dots
\end{align}
\setcounter{equation}{0}
\clearpage

%%%%%%%%%%%%%%%%%%%%%%%%% EXERCÍCIOS DE CURVAS PLANAS %%%%%%%%%%%%%%%%%%%%%%%%%%

\section{2. Curvas Planas}

%%%%%%%%%%%%%%%%%%%%%%%%%%%%%%%%% EXERCÍCIO 1 %%%%%%%%%%%%%%%%%%%%%%%%%%%%%%%%%%

\subsection{1.}

Desenhe as images das seguintes curvas, indicando o sentido do percurso:

\subsubsection{(a)}
$\gamma(t) = (1, t), t \in \mathbb{R}$
\begin{center}
  \begin{tikzpicture}
    \begin{axis}[
      width = 0.75\textwidth,
      axis x line = center,
      axis y line = center,
      axis line style={shorten >= -10pt, thick},
      xlabel = {$x$},
      ylabel = {$y$},
      x label style = {
        at = {(current axis.right of origin)},
        anchor = west,
        xshift = 10pt
      },
      y label style = {
        at = {(current axis.above origin)},
        anchor = south,
        yshift = 10pt
      },
      xtick distance = 1,
      ytick distance = 1,
      minor tick num = 4,
      xmin = -2, xmax = 2,
      ymin = -2, ymax = 2,
      grid = both,
      major grid style = {gray},
      minor grid style = {gray!25},
    ]
    \addplot[
      red,
      thick,
      ->,
      domain = -2 : 2,
      variable = t,
    ] ({1}, {t});
    \end{axis}
  \end{tikzpicture}
\end{center}
\clearpage

\subsubsection{(b)}
$\gamma(t) = (cos^2 \, t, sin \, t), 0 \leq t \leq 2 \pi$
\begin{center}
  \begin{tikzpicture}
    \begin{axis}[
      width = 0.75\textwidth,
      axis x line = center,
      axis y line = center,
      axis line style = {shorten >= -10pt, thick},
      xlabel = {$x$},
      ylabel = {$y$},
      x label style = {
        at = {(current axis.right of origin)},
        anchor = west,
        xshift = 10pt
      },
      y label style = {
        at = {(current axis.above origin)},
        anchor = south,
        yshift = 10pt
      },
      xtick distance = 1,
      ytick distance = 1,
      minor tick num = 4,
      xmin = -2, xmax = 2,
      ymin = -2, ymax = 2,
      grid = both,
      major grid style = {gray},
      minor grid style = {gray!25},
      trig format plots = rad,
    ]
    \addplot[
      red,
      thick,
      ->,
      samples = 500,
      domain = 0 : 2*pi,
      variable = t,
    ] ({cos(t)^2}, {sin(t)});
    \end{axis}
  \end{tikzpicture}
\end{center}
\clearpage

\subsubsection{(c)}
$\gamma(t) = (sin \, t, sin^2 \, t), t \in \mathbb{R}$
\begin{center}
  \begin{tikzpicture}
    \begin{axis}[
      width = 0.75\textwidth,
      axis x line = center,
      axis y line = center,
      axis line style = {shorten >= -10pt, thick},
      xlabel = {$x$},
      ylabel = {$y$},
      x label style = {
        at = {(current axis.right of origin)},
        anchor = west,
        xshift = 10pt
      },
      y label style = {
        at = {(current axis.above origin)},
        anchor = south,
        yshift = 10pt
      },
      xtick distance = 1,
      ytick distance = 1,
      minor tick num = 4,
      xmin = -2, xmax = 2,
      ymin = -2, ymax = 2,
      grid = both,
      major grid style = {gray},
      minor grid style = {gray!25},
      trig format plots = rad,
    ]
    \addplot[
      red,
      thick,
      >-<,
      samples = 500,
      domain = pi/2 : 3 * pi/2,
      variable = t,
    ] ({sin(t)}, {sin(t)^2});
    \end{axis}
  \end{tikzpicture}
\end{center}
\clearpage

\subsubsection{(d)}
$\gamma(t) = (2 + cos \, t, 3 + 4 \, sin \, t), t \in [-\pi, \pi]$
\begin{center}
  \begin{tikzpicture}
    \begin{axis}[
      width = 0.75\textwidth,
      axis x line = center,
      axis y line = center,
      axis equal,
      axis line style = {shorten >= -10pt, thick},
      xlabel = {$x$},
      ylabel = {$y$},
      x label style={
        at = {(current axis.right of origin)},
        anchor = west,
        xshift = 10pt
      },
      y label style = {
        at = {(current axis.above origin)},
        anchor = south,
        yshift = 10pt
      },
      xtick distance = 1,
      ytick distance = 1,
      minor tick num = 4,
      ymin = -1, ymax = 7,
      grid = both,
      major grid style = {gray},
      minor grid style = {gray!25},
      trig format plots = rad,
    ]
    \addplot[
      red,
      thick,
      ->,
      samples = 500,
      domain = -pi : pi,
      variable = t,
    ] ({2 + cos(t)}, {3 + 4*sin(t)});
    \end{axis}
  \end{tikzpicture}
\end{center}
\clearpage

\subsubsection{(e)}
$\gamma(t) = (\frac{1}{2}, 1 - t), t \in [-2, 0]$
\begin{center}
  \begin{tikzpicture}
    \begin{axis}[
      width = 0.75\textwidth,
      axis y line = center,
      axis x line = center,
      axis line style = {shorten >= -10pt, thick},
      xlabel = {$x$},
      ylabel = {$y$},
      x label style = {
        at = {(current axis.right of origin)},
        anchor = west,
        xshift = 10pt
      },
      y label style = {
        at = {(current axis.above origin)},
        anchor = south,
        yshift = 10pt
      },
      xtick distance = 1,
      ytick distance = 1,
      minor tick num = 4,
      xmin = -2, xmax = 2,
      ymin = 0, ymax = 4,
      grid = both,
      major grid style = {gray},
      minor grid style = {gray!25},
    ]
    \addplot[
      red,
      thick,
      ->,
      samples = 500,
      domain = -2:0,
      variable = t,
    ] ({1/2}, {1 - t});
    \end{axis}
  \end{tikzpicture}
\end{center}
\clearpage

\subsubsection{(f)}
$\gamma(t) = (e^t cos \, t, e^t sin \, t), t \geq 0$
\begin{center}
  \begin{tikzpicture}
    \begin{axis}[
      width = 0.75\textwidth,
      axis x line = center,
      axis y line = center,
      axis equal,
      axis line style = {shorten >= -10pt, thick},
      xlabel = {$x$},
      ylabel = {$y$},
      x label style = {
        at = {(current axis.right of origin)},
        anchor = west,
        xshift = 10pt
      },
      y label style = {
        at = {(current axis.above origin)},
        anchor = south,
        yshift = 10pt
      },
      xtick distance = 50,
      ytick distance = 50,
      minor tick num = 4,
      grid = both,
      major grid style = {gray},
      minor grid style = {gray!25},
      trig format plots = rad,
    ]
    \addplot[
      red,
      thick,
      ->,
      samples = 500,
      domain = 0:5,
      variable = t,
    ] ({e^t * cos(t)}, {e^t * sin(t)});
    \end{axis}
  \end{tikzpicture}
\end{center}
\clearpage

\subsubsection{(g)}
$\gamma(t) = (sec \, t, tan \, t), t \in \left]-\frac{\pi}{2}, \frac{\pi}{2}\right[$
\begin{center}
  \begin{tikzpicture}
    \begin{axis}[
      width = 0.75\textwidth,
      axis x line = center,
      axis y line = center,
      axis equal,
      axis line style = {shorten >= -10pt, thick},
      xlabel = {$x$},
      ylabel = {$y$},
      x label style = {
        at = {(current axis.right of origin)},
        anchor = west,
        xshift = 10pt
      },
      y label style = {
        at = {(current axis.above origin)},
        anchor = south,
        yshift = 10pt
      },
      xtick distance = 1,
      ytick distance = 1,
      minor tick num = 4,
      grid = both,
      major grid style = {gray},
      minor grid style = {gray!25},
      restrict expr to domain = {x}{0:4},
      trig format plots = rad,
    ]
    \addplot[
      red,
      thick,
      ->,
      samples = 500,
      domain = -pi/2 : pi/2,
      variable = t,
    ] ({sec(t)}, {tan(t)});
    \end{axis}
  \end{tikzpicture}
\end{center}
\clearpage

\subsubsection{(h)}
$\gamma(t) = (\sqrt{2} \, cos \, t, 2 \, sin \, t), t \in \mathbb{R}$
\begin{center}
  \begin{tikzpicture}
    \begin{axis}[
      width = 0.75\textwidth,
      axis x line = center,
      axis y line = center,
      axis line style = {shorten >= -10pt, thick},
      xlabel = {$x$},
      ylabel = {$y$},
      x label style = {
        at = {(current axis.right of origin)},
        anchor = west,
        xshift = 10pt
      },
      y label style = {
        at = {(current axis.above origin)},
        anchor = south,
        yshift = 10pt
      },
      xtick distance = 1,
      ytick distance = 1,
      minor tick num = 4,
      xmin = -2, xmax = 2,
      ymin = -2, ymax = 2,
      grid = both,
      major grid style = {gray},
      minor grid style = {gray!25},
      trig format plots = rad,
    ]
    \addplot[
      red,
      thick,
      ->,
      samples = 500,
      domain = 0 : 2*pi,
      variable = t,
    ] ({sqrt(2) * cos(t)}, {2*sin(t)});
    \end{axis}
  \end{tikzpicture}
\end{center}
\clearpage

\subsubsection{(i)}
$\gamma(t) = (sin \, t, cos^2 \, t + 2), t \in \mathbb{R}$
\begin{center}
  \begin{tikzpicture}
    \begin{axis}[
      width = 0.75\textwidth,
      axis x line = center,
      axis y line = center,
      axis line style = {shorten >= -10pt, thick},
      xlabel = {$x$},
      ylabel = {$y$},
      x label style = {
        at = {(current axis.right of origin)},
        anchor = west,
        xshift = 10pt
      },
      y label style = {
        at = {(current axis.above origin)},
        anchor = south,
        yshift = 10pt
      },
      xtick distance = 1,
      ytick distance = 1,
      minor tick num = 4,
      xmin = -2, xmax = 2,
      ymin = -1, ymax = 3,
      grid = both,
      major grid style = {gray},
      minor grid style = {gray!25},
      trig format plots = rad,
    ]
    \addplot[
      red,
      thick,
      >-<,
      samples = 500,
      domain = -pi/2 : pi/2,
      variable = t,
    ] ({sin(t)}, {cos(t)^2 + 2});
    \end{axis}
  \end{tikzpicture}
\end{center}
\clearpage
\end{document}
